\documentclass[aspectratio=43]{beamer}
\usepackage[spanish]{babel}
\usepackage[utf8]{inputenc}

\title{Android botnets for multi targeted attacks}
\author{Jaime García González y Adán Cano Moreno}
\institute{Universidad Carlos III de Madrid}

  
%\usetheme{Madrid}
%\usetheme{CambridgeUS}

% si quieres que metropolis use una tipografía TrueType (Fira Sans), debes
% compilar con lualatex
\usetheme{metropolis}

% descomentar si usas un tema distinto de 'metropolis'
%\AtBeginSection{
% \begin{frame}{Contenido}
%    \tableofcontents[currentsection]
%  \end{frame}
%}
\setbeamercovered{transparent}

\begin{document}

\maketitle
\begin{frame}{Contenido}
  \tableofcontents
\end{frame}

\section{Información del artículo}
\begin{frame}{Valentin Hamon}
  \begin{itemize}
   \item  Ingeniero de Seguridad informática de Sistemas por la ESIEA.
  \item En ese momento: Estudiante e investigador en ESIEA. 
  \item Actualidad:  FAMOCO
  \item Otros artículos: \emph{Malicious URI Resolving in
PDFs} (muestra como una petición HTTP desde un PDF puede
ser una buena herramienta para un atacante)
  \end{itemize}
\end{frame}
\begin{frame}{Springer-Verlag}
  \begin{itemize}
   \item  Filial francesa
de Springer, fundada en 1986.
  \item Ámbito: medicina, matemáticas, estadística, informática e ingeniería.
  \item Otras actividades:  publicación de libros del campo de las ciencias y editorial Springer en
Francia.

  \end{itemize}
\end{frame}
\section{Contenido del artículo}

\begin{frame}{Introducción}
\begin{itemize}
\item Los dispositivos móviles tienen muchos sensores que son atractivos para
los atacantes --> acceso a través de las aplicaciones.

\item Prescripción del procedimiento a seguir para realizar un ataque botnet 
sobre distintos dispositivos móviles Android al mismo tiempo para capturar información. 

\item Objetivo: exponer el potencial que tiene este tipo de ataques y el peligro
que puede suponer. 

\item Ejemplo de ataque: mostrar la localización de diferentes dispositivos móviles a través de diferentes fases. 

  \end{itemize}  
\end{frame}

\begin{frame}{Fases}
  El artículo divide el procedimiento de ataque en 5 fases:   
  \begin{itemize}
  \item Recoger información
  \item Almacenamiento y gestión de la información
  \item Mostrar la información
  \item Información de verificación
  \item Determinar puntos de encuentro entre dispositivos
  \end{itemize}  
\end{frame}

\begin{frame}{Fases: Recoger información}
  
  \begin{itemize}
  \item Ataque botnet: cualquier grupo de dispositivos infectados y controlados por un atacante de forma remota.  
  
  \item Usos: ataques de denegación de servicio distribuidos (DDoS), envío de Spam, minería de Bitcoins y robo de Bitcoins.  


 \item Ejemplos: Conficker, Zeus, Waledac, Mariposa y Kelihos. 
 
  \end{itemize}
  
  
\begin{figure}[htbp] 	
   \includegraphics[width=0.50\textwidth]{figuras/botnet}  
  \label{botnet}

\end{figure}

\end{frame}

\begin{frame}{Fases: Recoger información}
  
  \begin{itemize}
  
  \item Se necesita que el teléfono esté encendido el mayor tiempo posible.   
  
  \item Se necesita una red de bots móvil para enviar datos de geolocalización
al servidor --> se crea una instancia de nuestra clase de LocationListener al comenzar la actividad principal.

  \item Se genera un protocolo para enviar los datos recogidos en una sola cadena. 
  
  \item Es posible transferir datos de dos maneras diferentes
recogidos de los smartphones a un servidor --> HTTP o SMS.

  \end{itemize}
\end{frame}

\begin{frame}{Fases: Almacenamiento y gestión de la información}

\begin{itemize}
  
  \item PHP o MySQL para gestionar la información --> instalados por defectos en la red de servidores mundial.   

  \item Se concatenan todas las variables a almacenar en una única cadena (localización, fecha\ldots). 
  
  \item Primera idea para almacenar los datos: crear para cada botnet una nueva tabla con el IMEI (\emph{International Mobile Equipment Identity}) como nombre --> problema cuando intentamos crear una nueva tabla con un número como nombre (no permitido en MySQL).
  
  \item Solución: traducimos los números de IMEI en caracteres (0 = a, 1 = b, \ldots).  IMEI es único -->  n tablas con n nombres diferentes.

  \end{itemize}
  
\end{frame}

\begin{frame}{Fases: Mostrar la información}

\begin{itemize}

\item Incluir el encabezado de una web HTML con algunas líneas para inicializar la API de Google maps.

\item Se inicializa una matriz para tener diferentes colores para nuestros marcadores en el mapa.

\item Consulta SELECT a los IMEIs. Usamos \emph{fecth()} de PHP para acceder a cada línea de la tabla y almacenamos los nombres de tablas en un array para poder hacer consultas en todas las tablas de botnets.


	\begin{figure}[htbp] 
	\begin{flushright} 
   \includegraphics[width=0.45\textwidth]{figuras/maps}  
  \label{maps}
  \end{flushright}
\end{figure}


\end{itemize}
  
\end{frame}

\begin{frame}{Fases: Información de verificación}

\begin{itemize}
\item Determinar el comportamiento de las personas infectadas --> si las víctimas están en un autobús, en un tren, andando o paradas. Tenemos que determinar la velocidad de movimiento de nuestras víctimas.

\item Problema: recopilar suficientes datos para determinar el comportamiento de la víctima.

\item Antes de analizar cualquier comportamiento se debe comprobar los datos. Se crea una nueva tabla con un nombre que es la concatenación de los nombres de las n tablas separados por "\_". 

\item La fecha será un atributo único en la tabla (sólo tenemos una geolocalización entrada de datos en la base de datos por segundo).


\end{itemize}
  
\end{frame}


\begin{frame}{Fases: Información de verificación}

\begin{itemize}
\item Podemos separar los datos de todas nuestras redes de bots por cualquier período de tiempo.

\item Podemos mostrar en el mapa datos de nuestras redes de bots con respecto a una fecha precisa.

\item Podemos seleccionar y mostrar datos de múltiples objetivos con criterios muy específicos (a través de la fecha.
\end{itemize}
  	\begin{figure}[htbp] 

   \includegraphics[width=0.45\textwidth]{figuras/IMEIsConcatenados}  
  \label{IMEIsConcatenados}

\end{figure}

\end{frame}

\begin{frame}{Fases: Determinar puntos de encuentro entre dispositivos}
  
\end{frame}



\maketitle

\end{document}
